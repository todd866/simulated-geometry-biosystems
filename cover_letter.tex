\documentclass[11pt]{letter}
\usepackage[margin=1in]{geometry}
\usepackage{hyperref}

\signature{Ian Todd\\Sydney Medical School\\University of Sydney}
\address{Sydney Medical School\\University of Sydney\\Sydney, NSW 2006, Australia\\itod2305@uni.sydney.edu.au}

\begin{document}

\begin{letter}{Editorial Office\\BioSystems}

\opening{Dear Editors,}

I am pleased to submit ``Simulated Geometry: Why Information Alone Cannot Be Alive'' for consideration as a hypothesis/perspective article in \textit{BioSystems}.

\textbf{The central question:} Large language models pass examinations and solve complex problems, yet something appears to be missing. We propose a measurable criterion distinguishing \emph{simulated relational structure} from \emph{endogenously maintained constraint manifolds}---grounded in dynamical systems theory rather than contested philosophical categories.

\textbf{The proposal:} Life instantiates geometry; AI simulates geometry using information. \emph{Information} is what can be copied without loss (Shannon entropy, bits). \emph{Geometry} is the constrained manifold within which a system operates---the dimensionality of knowing, not just the quantity of known. Current AI has vast information but minimal self-maintained geometry.

\textbf{Key contributions:}
\begin{itemize}
\item A mathematical framework distinguishing information from geometry, with implications for biological organization
\item Formalization via \emph{match error}: agreement among independent complexity estimators as the signature of genuine geometry
\item Biology as existence proof: cross-scale coherence, constraint closure, and dormancy as geometry-without-dynamics
\item Analysis of why current AI simulates but does not instantiate geometry (no intrinsic oscillations, no viability coupling)
\item Testable predictions and experimental directions
\end{itemize}

\textbf{Relevance to BioSystems:} The paper addresses foundational questions about biological information and self-organization---core themes of the journal. The analysis connects to quantum-like formalisms in biological dynamics (contextuality, non-commuting operations) while remaining agnostic about specific microphysical mechanisms.

The manuscript has not been submitted elsewhere and presents original synthesis.

\closing{Sincerely,}

\end{letter}
\end{document}
